%%%%%%%%%%%%%%%%%%%%%%%%%%%%%%%%%%%%%%%%%
% Journal Article
% LaTeX Template
% Version 1.4 (15/5/16)
%
% This template has been downloaded from:
% http://www.LaTeXTemplates.com
%
% Original author:
% Frits Wenneker (http://www.howtotex.com) with extensive modifications by
% Vel (vel@LaTeXTemplates.com)
%
% License:
% CC BY-NC-SA 3.0 (http://creativecommons.org/licenses/by-nc-sa/3.0/)
%
%%%%%%%%%%%%%%%%%%%%%%%%%%%%%%%%%%%%%%%%%

%----------------------------------------------------------------------------------------
%	PACKAGES AND OTHER DOCUMENT CONFIGURATIONS
%----------------------------------------------------------------------------------------

\documentclass[twoside,twocolumn]{article}

\usepackage{blindtext} % Package to generate dummy text throughout this template 

\usepackage[sc]{mathpazo} % Use the Palatino font
\usepackage[T1]{fontenc} % Use 8-bit encoding that has 256 glyphs
\linespread{1.05} % Line spacing - Palatino needs more space between lines
\usepackage{microtype} % Slightly tweak font spacing for aesthetics

\usepackage[polish]{babel} % Language hyphenation and typographical rules

\usepackage[hmarginratio=1:1,top=32mm,columnsep=20pt]{geometry} % Document margins
\usepackage[hang, small,labelfont=bf,up,textfont=it,up]{caption} % Custom captions under/above floats in tables or figures
\usepackage{booktabs} % Horizontal rules in tables

\usepackage{lettrine} % The lettrine is the first enlarged letter at the beginning of the text

\usepackage{enumitem} % Customized lists
\setlist[itemize]{noitemsep} % Make itemize lists more compact

\usepackage{abstract} % Allows abstract customization
\renewcommand{\abstractnamefont}{\normalfont\bfseries} % Set the "Abstract" text to bold
\renewcommand{\abstracttextfont}{\normalfont\small\itshape} % Set the abstract itself to small italic text

\usepackage{titlesec} % Allows customization of titles
\renewcommand\thesection{\Roman{section}} % Roman numerals for the sections
\renewcommand\thesubsection{\roman{subsection}} % roman numerals for subsections
\titleformat{\section}[block]{\large\scshape\centering}{\thesection.}{1em}{} % Change the look of the section titles
\titleformat{\subsection}[block]{\large}{\thesubsection.}{1em}{} % Change the look of the section titles

\usepackage{fancyhdr} % Headers and footers
\pagestyle{fancy} % All pages have headers and footers
\fancyhead{} % Blank out the default header
\fancyfoot{} % Blank out the default footer
\fancyhead[C]{Running title $\bullet$ May 2016 $\bullet$ Vol. XXI, No. 1} % Custom header text
\fancyfoot[RO,LE]{\thepage} % Custom footer text

\usepackage{titling} % Customizing the title section

\usepackage{hyperref} % For hyperlinks in the PDF

%----------------------------------------------------------------------------------------
%	TITLE SECTION
%----------------------------------------------------------------------------------------

\setlength{\droptitle}{-4\baselineskip} % Move the title up

\pretitle{\begin{center}\Huge\bfseries} % Article title formatting
\posttitle{\end{center}} % Article title closing formatting
\title{Algorytm Decyzyjny dla Rummikub} % Article title
\author{%
\textsc{Aleksander Strzelecki}\\[1ex] % Your name
\normalsize Politechnika Gdańska \\ % Your institution
\normalsize \href{mailto:s179971@student.pg.edu.pl}{s179971@student.pg.edu.pl} % Your email address
}
\date{\today} % Leave empty to omit a date
\renewcommand{\maketitlehookd}{%
\begin{abstract}
\noindent
% Rummikub jest grą dla 2-4 osób. Gra składa się z 106 kości. Każda kość posiada numer od 1 do 13
% oraz jeden z kolorów(czarny, niebieski, pomarańczowy, czerwony). Każda kość występuje w dwóch egzemplarzach.
% Czyli mamy 13*4*2+2(jokery) = 106 kości. Joker może zastępować dowolną kość. Gracze zaczynają grę 
% z losowymi 14 koścmi. Reszta kości jest pulą, z której trzeba dobierać kości w przypadku braku ruchu przez 
% gracza. Celem gracza jest układanie kości na stole w serie (np. 1(czarny),2(czarny),3(czarny)) lub w grupy
% (1(czarny),1(czerwony),1(niebieski)). Seria lub grupa nie może liczyć mniej niż 3 kości. Gracz podczas ruchu może manipulować
% również aktualnymi seriami i grupami dostępnymi na stole. Gracz musi wykonać otwarcie (wyłożenie kości o sumie wartości 
% równej 30) w celu rozpoczęcia rozgrywki. Celem gry jest pozbycie się przez Gracza wszystkich swoich kości.
\end{abstract}
}

%----------------------------------------------------------------------------------------

\begin{document}

% Print the title
\maketitle

%----------------------------------------------------------------------------------------
%	ARTICLE CONTENTS
%----------------------------------------------------------------------------------------

\section{Wstęp}

% \lettrine[nindent=0em,lines=3]{L} orem ipsum dolor sit amet, consectetur adipiscing elit.


%------------------------------------------------

\section{Prace powiązane}

Algorytm decyzyjny przedstawiony w \cite{Rijn:2016fr} nie zakładał możliwości 
manipulowania istniejącymi płytkami. Algorytm polegał na wyliczeniu ocen ruchów w zależności od 
płytki rozpoczynającej ruch. Następnie wybierany został zestaw ruchów, który dawał maksymalną 
ocenę. Schemat szukania ruchów polegał na przemiennym tworzeniu ciągów i grup z dostępnych płytek.

W artykule \cite{Slagle:1963js} zostało przedstawione rozwiązywanie całki symbolicznie.
Autor podzielił przekształcenia na korzystne i heurystyczne. Przekstałcenia korzystne wykonywane są pierwsze, a
heurystyczne w przypadku braku możliwości przekształcenia korzystnego. W miejscach, gdzie możliwe jest kilka 
przekształceń otrzymujemy rozgałęzienie. Następnie sprawdzana jest złożoność wyrażeń w otrzymanych gałęziach i 
do dalszego rozwiązywania wybierana jest zawsze ta o najmniejszej złożoności.
%------------------------------------------------
\section{Proponowane rozwiązanie}
%------------------------------------------------
\section{Wyniki}
%------------------------------------------------

\section{Podsumowanie}

%----------------------------------------------------------------------------------------
%	REFERENCE LIST
%----------------------------------------------------------------------------------------

\begin{thebibliography}{99} % Bibliography - this is intentionally simple in this template

\bibitem{Rijn:2016fr}
Fvan Rijn, Jan \& Takes, Frank \& Vis, Jonathan. (2016). 
\newblock The Complexity of Rummikub Problems.
\newblock {\em Leiden Institute of Advanced Computer Science, Leiden University, The Netherlands}.
 
\bibitem{Slagle:1963js}
James R. Slagle. 1963.
\newblock A Heuristic Program that Solves Symbolic Integration Problems in Freshman Calculus. 
\newblock {\em J. ACM 10, 4 (Oct. 1963), 507–520.}
\end{thebibliography}

%----------------------------------------------------------------------------------------

\end{document}
